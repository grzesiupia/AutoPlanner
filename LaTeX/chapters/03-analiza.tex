
\chapter{Analiza i porównanie możliwych rozwiązań}
\section{Analiza problemu}
\section{Aktualnie dostępne rozwiązania}
\subsection{aSc TimeTables}
aSc TimeTables~\cite{asc} to aplikacja desktopowa wspomagająca przygotowywanie planów zajęć. Narzędzie umożliwia generowanie planów na podstawie zdefiniowanych wymagań, wprowadzenie do nich ręcznych poprawek oraz wyszukiwanie konfliktów we wprowadzonych zmianach. aSc TimeTables jest najbardziej rozbudowanym rozwiązaniem tego typu dostępnym na rynku pozwalającym na tworzenie planów zajęć dla szkół i uczelni. Do dodatkowych funkcji programu należy możliwość importu danych z pliku, zdolność mapowania szkoły oraz udostępnienia planów uczniom i nauczycielom za pomocą aplikacji mobilnej. Z wszechstronnością i bogactwem funkcji wiąże się wysoki poziom umiejętności potrzebny do poprawnego wykorzystania aplikacji. Do pozostałych wad programu należy brak regularnych aktualizacji, podatność na błędy w generacji planu, wysoka cena oraz dostępność ograniczona do systemu Windows.
\subsection{Prime Timetable}
Prime Timetables~\cite{prime} to aplikacja internetowa przeznaczona dla organizacji edukacyjnych umożliwiająca zarówno ręczne jak i automatyczne układanie planów lekcji. Prime Timetables pozwala na wspólne tworzenie planów przez kilku użytkowników oraz udostępnianie gotowych planów dla uczniów i nauczycieli posiadających konta w serwisie. Aplikacja posiada rozbudowany zestaw narzędzi umożliwiających określanie ograniczeń związanych z automatyczną generacją planu. Główną wadą rozwiązania jest wysoka opłata miesięczna, której wysokość dodatkowo zależy od liczby nauczycieli w szkole. 
\subsection{SuperSaas}
SuperSaas~\cite{saas} to program do zarządzania szkołami i innymi instytucjami, którego głównym atutem jest wbudowany system rezerwacji. Przy pomocy konta WordPress użytkownicy aplikacji mogą umawiać terminy wizyt, a także dokonywać za nie płatności. SuperSaas cechuje niska cena oraz dostępność z poziomu przeglądarki. Duża część funkcjonalności aplikacji nie jest przeznaczona dla szkół. Pomimo możliwości wspomagania ręcznego układania planów zajęć, program nie pozwala na automatyczną ich generację, ani nawet wykrywanie konfliktów. 
\section{Możliwe podejścia}