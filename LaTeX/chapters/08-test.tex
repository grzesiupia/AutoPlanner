
\chapter{Testowanie}
\section{Frontend}
Testy w części frondendowej aplikacji wykonywane są przy pomocy platformy programistycznej Jest. W projekcie można wyróżnić podział na dwa typy testów. Pierwszym typem są testy spójności magazynu Vuex. Przykładowym testem tego typu jest test przypisanie tokenu JWT~(zob.~listing~\ref{lst:vuextest}).
\begin{lstlisting}[caption=Test spójności magazynu Vuex,label={lst:vuextest}] 
describe('mutations', () => {
  test('setToken', () => {
    const token = "exampletoken"
    const state = {
      token: ""
    }
    store.commit('SET_TOKEN', { token })
    expect(store.getters.getToken.token).toBe(token)
  })
\end{lstlisting}

Drugim typem testów są testy interfejsu użytkownika. Przykładem takiego testu jest test przycisku, który odpowiada za zwiększanie liczby prowadzonych przez nauczyciela przedmiotów~(zob.~listing~\ref{lst:uitest}).
\begin{lstlisting}[caption=Test interfejsu użytkownika,label={lst:uitest}] 
describe('userInput', () => {
  test('addSubject', async () => {
    const wrapper = mount(Step2)
    const button = wrapper.find('addSubjectButton')

    expect(subjectNumber).toBe(0)
    await button.trigger('click')
    expect(subjectNumber).toBe(1)
})})
\end{lstlisting}
\section{Backend}
\section{Algorytm}

