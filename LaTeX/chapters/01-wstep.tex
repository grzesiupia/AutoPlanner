
\chapter{Wstęp}
(Źródła?)Tematem podjętym w pracy jest aplikacja służąca do generowania planów zajęć. Główną motywacją do podjęcia takiego (tego?,"") tematu stanowią wady obecnie stosowanego przez większość szkół manualnego tworzenia planów zajęć. Ręczne tworzenie planu jest czasochłonne i wymaga dużego nakładu pracy. Dla osób odpowiedzialnych za ich przygotowanie (dalej zwanymi planistami) jest to zadanie monotonne, a także przytłaczające. Planiści, nawet ci z dużym doświadczeniem, nie są zdolni do utworzenia planu, który optymalnie wykorzystywałby godziny uczniów, nauczycieli, a także dostępność sali lekcyjnych. Skutkuje to znaczną liczbą niewykorzystanego czasu w środku dnia lekcyjnego.

Celem pracy jest zaprojektowanie aplikacji, dzięki której po podaniu niezbędnych danych, możliwe byłoby automatyczne wygenerowanie planu zajęć dla szkoły. Aplikacja ma umożliwić planiście dodawanie danych o przedmiotach, nauczycielach, salach i klasach. Na podstawie podanych danych planista ma mieć możliwość generacji rozkładu zajęć dla wszystkich klas w szkole. Aplikacja ma być przeznaczona dla szkół podstawowych oraz średnich. Ograniczenie to wynika z założenia niepodzielności klasy. W przypadku uczelni wyższych niejednolity podział na grupy znacząco zwiększa poziom skomplikowania rozwiązywanego problemu. 

Projekt można podzielić na pięć głównych części: konfigurację infrastruktury informatycznej, implementację back-end, implementację front-end, implementację algorytmu oraz testy.

Praca ma następującą strukturę. Rozdział drugi poświecony jest podstawom teoretycznym. Rozdział trzeci zawiera analizę problemu i dostępnych rozwiązań. Rozdział czwarty to opis infrakstruktury informatycznej. Rozdział piąty omawia część fronendową aplikacji. Rozdział szósty charakteryzuje backend aplikacji. Rozdział siódmy wyjaśnia działanie algorytmu generacji planu. Rozdział ósmy opisuje testy. Rozdział dziewiąty stanowią wnioski. Rozdział dziesiąty jest podumowaniem pracy. 

Implementacja aplikacji została wykonana przez cztery osoby.
Mateusz Biernacki wykonał ...
Dominik Boła wykonał ...
Maciej Goral wykonał ...
Grzegorz Piątkowski wykonał ...

Wstęp do pracy powinien zawierać następujące elementy:
\begin{itemize}
    \item krótkie uzasadnienie podjęcia tematu; 
    \item cel pracy (patrz niżej); 
    \item zakres (przedmiotowy, podmiotowy, czasowy) wyjaśniający, w jakim rozmiarze praca będzie realizowana; 
    \item ewentualne hipotezy, które autor zamierza sprawdzić lub udowodnić; 
    \item krótką charakterystykę źródeł, zwłaszcza literaturowych; 
    \item układ pracy (patrz niżej), czyli zwięzłą charakterystykę zawartości poszczególnych rozdziałów; 
    \item ewentualne uwagi dotyczące realizacji tematu pracy np.~trudności, które pojawiły się w trakcie 
    realizacji poszczególnych zadań, uwagi dotyczące wykorzystywanego sprzętu, współpraca z firmami zewnętrznymi. 
\end{itemize}

\noindent
\textbf{Wstęp do pracy musi się kończyć dwoma następującymi akapitami:}
\begin{quote}
Celem pracy jest opracowanie / wykonanie analizy / zaprojektowanie / ...........
\end{quote}
oraz:
\begin{quote}
Struktura pracy jest następująca. W rozdziale 2 przedstawiono przegląd literatury na temat ........ 
Rozdział 3 jest poświęcony ....... (kilka zdań). 
Rozdział 4 zawiera ..... (kilka zdań) ............ itd. 
Rozdział X stanowi podsumowanie pracy. 
\end{quote}

W przypadku prac inżynierskich zespołowych lub magisterskich 2-osobowych, po tych dwóch w/w akapitach 
musi w pracy znaleźć się akapit, w którym będzie opisany udział w pracy poszczególnych członków zespołu. Na przykład:

\begin{quote}
Jan Kowalski w ramach niniejszej pracy wykonał projekt tego i tego, opracował ......
Grzegorz Brzęczyszczykiewicz wykonał ......, itd. 
\end{quote}

