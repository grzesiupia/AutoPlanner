
\chapter{Wstęp}
W dzisiejszych czasach ciężko znaleźć osobę która nigdy nie korzystała z internetu. Większość naszego społeczeństwa używa go na porządku dziennym. Wykorzystywany jest prawie w każdej dziedzinie życia, służyć może np. do komunikacji w czasie rzeczywistym, dokonywania szybkich płatności, zakupów internetowych, nauki czy też pracy. Przykłady można wymieniać bez końca, jednak trzeba zwrócić szczególną uwagę na narzędzia, dzięki którym możemy korzystać z sieci w tak szerokim zakresie. Jednym z najpopularniejszych wykorzystywanych oraz dynamicznie rozwijanych rozwiązań są aplikacje webowe. Jedną z ich głównych zalet jest możliwość korzystania z nich niezależnie od używanego urządzenia, o ile ma ono dostęp do internetu oraz posiada przeglądarkę internetową. W przeciwieństwie do aplikacji desktopowych, wszelkie aktualizację są dokonywane przez administratora, co w kontekście rozwoju oprogramowania jest bardzo wygodne zarówno dla programisty jak i użytkownika. Są to jedne z wielu zalet, które z pewnością przyczyniły się do szybkiego rozwoju aplikacji internetowych oraz związanych z nimi technologii.

Tematem podjętym w pracy jest aplikacja służąca do generowania planów zajęć. Zaprojektowanie tego typu rozwiązania, daje możliwość nauki rozmaitych technologii informatycznych, powszechnie wykorzystywanych w praktyce. Główną motywacją do podjęcia takiego tematu stanowią wady obecnie stosowanego przez większość szkół manualnego tworzenia planów zajęć. Ręczne tworzenie planu jest czasochłonne i wymaga dużego nakładu pracy. Dla osób odpowiedzialnych za ich przygotowanie (dalej zwanych planistami) jest to zadanie monotonne, a także przytłaczające. Planiści, nawet ci z dużym doświadczeniem, nie są zdolni do utworzenia planu, który optymalnie wykorzystywałby godziny uczniów, nauczycieli, a także dostępność sali lekcyjnych. Skutkuje to znaczną liczbą niewykorzystanego czasu w środku dnia lekcyjnego.

Celem pracy jest zaprojektowanie aplikacji, dzięki której po podaniu niezbędnych danych, możliwe byłoby automatyczne wygenerowanie planu zajęć dla szkoły. Aplikacja ma umożliwić planiście dodawanie danych o przedmiotach, nauczycielach, salach i klasach. Na podstawie podanych danych planista ma mieć możliwość generacji rozkładu zajęć dla wszystkich klas w szkole. Aplikacja ma być przeznaczona dla szkół podstawowych oraz średnich. Ograniczenie to wynika z założenia niepodzielności klasy. W przypadku uczelni wyższych niejednolity podział na grupy znacząco zwiększa poziom skomplikowania rozwiązywanego problemu. 

Projekt można podzielić na cztery główe części: konfigurację infrastruktury informatycznej, implementację back-end, implementację front-end oraz implementację algorytmu.

Praca ma następującą strukturę. Rozdział drugi poświecony jest podstawom teoretycznym. Rozdział trzeci zawiera analizę problemu i dostępnych rozwiązań. Rozdział czwarty to opis metodyki pracy oraz infrakstruktury informatycznej. Rozdział piąty omawia część fronendową aplikacji. Rozdział szósty charakteryzuje backend aplikacji. Rozdział siódmy wyjaśnia działanie algorytmu generacji planu. Rozdział ósmy jest instrukcją użytkowania. Rozdział dziewiąty stanowią wnioski. 

Implementacja aplikacji została wykonana przez cztery osoby.
Mateusz Biernacki wykonał ...
Dominik Boła wykonał ...
Maciej Goral wykonał ...
Grzegorz Piątkowski wykonał ...
