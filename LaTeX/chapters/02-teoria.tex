% Co to jest NP - https://dbpedia.org/page/NP_(complexity)
% Co to jest problem NP-Trudny - https://www.baeldung.com/cs/p-np-np-complete-np-hard
% Ułozenie planu zajęć jako problem np-trudny - https://www.sciencedirect.com/science/article/pii/S1110016816000703#b0015


\chapter{Podstawy teoretyczne}
\section{Problem optymalizacyjny}
Problem optymalizacyjny~\cite{optimization_problem} jest to problem obliczeniowy, który polega na znalezieniu maksymalnej/minimalnej wartości pewnego parametru. Wartość takiego parametru zazwyczaj opisywana jest funkcją, dzięki której wartość parametru zależna jest od przeszukiwanych danych wejściowych. Jeśli poszukiwana jest jak najmniejsza wartość parametru, mówimy o problemie minimalizacyjnym i odpowiednio w przypadku poszukiwania największej wartości parametru, mówimy o problemie maksymalizacyjnym.

\section{Problem NP-trudny}
Problem NP-trudny~\cite{np_hard} jest problemem obliczeniowym, dla którego nie jest możliwym znalezienie rozwiązania w czasie wielomianowym przy wykorzystaniu niedeterministycznej maszyny Turinga, a sprawdzenie znalezionego rozwiązania jest co najmniej tak trudne jak każdego innego problemu z grupy NP. Problem optymalizacyjny jest jednym z problemów należących do grupy NP-trudnych.

\section{Heurystyka}
Heurystyka~\cite{heuristic} jest techniką rozwiązywania problemów w przypadku, gdy znalezienie dokładnego rozwiązania jest zbyt kosztowne. Metoda heurystyczna oferuje zmniejszenie kosztów rozwiązania problemu, jednak ceną takiego podejścia jest spadek dokładności rozwiązania czy nawet jego poprawności.  Przy wykorzystaniu metody heurystycznej otrzymanie optymalnego rozwiązania możliwe jest tylko w szczególnych przypadkach. Tego typu podejście wykorzystuje się również, w przypadku, gdy algorytm dokładny umożliwiający znalezienie rozwiązania optymalnego nie jest znany, w celu zawężenia pola badań.

\section{Algorytm ewolucyjny}
Algorytm ewolucyjny~\cite{evolution_algorithm} jest heurystycznym podejściem do rozwiązywania problemów, które nie mogą zostać rozwiązane w czasie wielomianowym, takie jak grupa problemów NP-trudnych, czy po prostu w celu zmniejszenia kosztów znalezienia rozwiązania problemu. Algorytmy ewolucyjne stosowane samodzielnie używane są zazwyczaj do rozwiązywania problemów optymalizacyjnych. Zastosowanie i działanie algorytmu ewolucyjnego jest bardzo proste do zrozumienia ze względu na to, że mamy do czynienia na co dzień z podobnym zjawiskiem w naturze czyli z selekcją naturalną. Przebieg działania algorytmu ewolucyjnego składa się z 4 głównych kroków.
	\begin{enumerate}
	\item \textbf{Inicjalizacja} -- W celu rozpoczęcia działania algorytmu, potrzebna jest pierwsza grupa rozwiązań (dalej nazywana populacją). Populacja zwierać będzie założoną liczbę możliwych rozwiązań (dalej nazywaną osobnikami). Zazwyczaj podczas inicjalizacji osobniki tworzone są w sposób losowy. Takie podejście jest wręcz zalecane, ponieważ umożliwia to przebadanie dużej różnorodności osobników, dzięki czemu finałowe rozwiązanie będzie lepsze.
	\item \textbf{Selekcja} -- Gdy pierwotna populacja jest gotowa, jej osobniki trzeba poddać ocenie. Funkcja oceny powinna składać się ze ściśle opisanych warunków opisujących środowisko, do którego osobniki muszą się przystosować. Im dokładniej środowisko zostanie opisane w funkcji oceny, tym lepsze będzie finalne rozwiązanie. Gdy funkcja jest poprawnie przygotowana, każdy z osobników musi zostać poddany ocenie, po której otrzymuje parametr oceny. Dzięki temu można wyróżnić rozwiązania lepsze od reszty. Z populacji zostaje wybrana założona liczba osobników o najwyższym parametrze oceny. Reszta osobników zostaje zabita.
	\item \textbf{Ewolucja} -- Ewolucja składa się z dwóch kroków: krzyżowania oraz mutacji. 
	\begin{enumerate}
		\item Krzyżowanie -- po otrzymaniu wybranych osobników z selekcji, użyte są one do stworzenia nowego pokolenia dla algorytmu, stając się osobnikami-rodzicami. Wykorzystując charakterystyki osobników-rodziców, utworzona zostaje populacja osobników-dzieci poprzez wymieszanie ze sobą charakterystyk osobników-rodziców. Po utworzeniu nowego pokolenia osobników-dzieci, osobniki-rodzice zostają zabite.
		\item Mutacja -- jest to prawdopodobnie najważniejszy krok ewolucji. Bez niego cała populacja bardzo szybko utknęłaby w miejscu, nie oferując żadnego sensownego rozwiązania. W tym kroku charakterystyka każdego osobnika-dziecka z nowego pokolenia poddana jest małym losowym zmianom w celu zróżnicowania ich od osobników-rodziców. Na końcu tego kroku osobniki-dzieci stają się nowym pokoleniem osobników w populacji, która może ponownie zostać poddana selekcji.
	\end{enumerate}
	\item \textbf{Finalizacja} -- Ostatecznie działanie algorytmu musi dobiec końca. W tym kroku z populacji zostaje wybrany osobnik z najwyższym parametrem oceny i zwrócony jako rozwiązanie. Są dwie możliwości, w których zakończenie działania algorytmu może zostać wywołane. Gdy osiągnie on maksymalny czas działania (np. założona maksymalna liczba pokoleń) lub gdy osiągnięty zostanie poszukiwany pułap parametru oceny.
	\end{enumerate}	


