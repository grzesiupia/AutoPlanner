% Co to jest NP - https://dbpedia.org/page/NP_(complexity)
% Co to jest problem NP-Trudny - https://www.baeldung.com/cs/p-np-np-complete-np-hard
% Ułozenie planu zajęć jako problem np-trudny - https://www.sciencedirect.com/science/article/pii/S1110016816000703#b0015


\chapter{Podstawy teoretyczne}
Wygenerowanie najlepszego możliwego planu zajęć dla dużej szkoły jest problemem NP-zupełnym. Liczba wszystkich możliwych do ułożenia poprawnych planów zajęć rośnie wykładniczo, wraz z wielkością szkoły, dla której plan jest tworzony. Nie jest możliwa, w rozsądnym przedziale czasowym, iteracja przez wszystkie rozwiązania i wybranie najlepszego z nich. Należy skorzystać z rozwiązania, które dostarczy rozwiązanie dobre, jak najbardziej zbliżone do optymalnego. Uzyskanie takiego wyniku umożliwia wykorzystanie podejścia ewolucyjnego.

Rozdział teoretyczny --- przegląd literatury naświetlający stan wiedzy na dany temat. 

Przegląd literatury naświetlający stan wiedzy na dany temat obejmuje rozdziały pisane na podstawie
literatury, której wykaz zamieszczany jest w części pracy pt.~\emph{Literatura} (lub inaczej \emph{Bibliografia},
\emph{Piśmiennictwo}). W tekście pracy muszą wystąpić odwołania do wszystkich pozycji zamieszczonych w
wykazie literatury. \textbf{Nie należy odnośników do literatury umieszczać w stopce strony.} Student jest
bezwzględnie zobowiązany do wskazywania źródeł pochodzenia informacji przedstawianych w pracy,
dotyczy to również rysunków, tabel, fragmentów kodu źródłowego programów itd. Należy także podać
adresy stron internetowych w przypadku źródeł pochodzących z Internetu.


