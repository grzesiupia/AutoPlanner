
\chapter{Projekt i implementacja strony serwerowej opartej na architekturze REST w technologi Django oraz bazy danych MySQL}
\section{Narzędzia i techonologie}
\subsection{Django}
Django~\cite{Django} jest darmową, wysoko poziomową platformą programistyczną przeznaczoną do tworzenia aplikacji internetowych. Dostarcza wiele narzędzi ułatwiających szybką oraz prostą implementacje. Oparta jest na wzorcu architektonicznym model-template-view. ~\ref{rys:django} Napisana jest w języku Python. Jednymi z najważniejszych cech Django są:
\begin{itemize}
	\item Łatwy i bezpieczny dostęp do bazy danych
	\item Duża skalowalność oraz wydajność
	\item Wbudowane zabezpieczenia przed popularnymi atakami
	\item Rozbudowana dokumentacja
\end{itemize}
\begin{figure}[H]
	\centering\includegraphics[width=\textwidth]{figures/DjangoSchemat}
	\caption{Schemat przepływu danych w Django~\cite{DjangoSchemat}}\label{rys:django}
\end{figure}
\subsection{MySQL}
MySQL~\cite{SQL} jest systemem służącym do zarządzania relacyjnymi bazami danych. Model relacyjny zapewnia łatwość w projektowaniu oraz implementacji. Udostępniony jest na licencji wolnego oprogramowania i dostępny jest dla wszystkich popularnych systemów operacyjnych. Bazy danych oparte na tym systemie są wstanie obsługiwać olbrzymie ilości zapytań w bardzo krótkim czasie.
\subsection{MySQL Workbench}
MySQL Workbench~\cite{Workbench} to narzędzie do projektowania, tworzenia oraz zarządzania bazami danych MySQL. Posiada bardzo przejrzysty i intuicyjny interfejs, przez co cieszy się dużą popularnością. Wiele podstawowych czynności takich jak np. tworzenie i edycja tabel, można wykonać bez znajomości zapytań SQL, gdyż są one generowane automatycznie.

\section{Implementacja serwera REST API}
\subsection{REST}
REST ~\cite{REST} jest stylem architektonicznym wprowadzającym pewien standard komunikacyjny dla internetowych systemów informatycznych. Jego najważniejszymi zaletami są szybkość oraz uniwersalność. Interfejsy programistyczne spełniające założenia REST mogą komunikować się dowolnym urządzeniem sieciowym, pod warunkiem wysyłania przez nie zapytań w odpowiednim formacie. Jednymi z głównych zasad tego stylu są:
\begin{itemize}
	\item Zastosowanie modelu klient-serwer
	\item Bezstanowość
	\item Wykorzystywanie pamięci cache przeglądarki w celu zapamiętywania odpowiedzi
	\item Interfejs programistyczny jednolity dla każdej aplikacji klienckiej
\end{itemize}
\begin{figure}[H]
	\centering\includegraphics[width=\textwidth]{figures/rest}
	\caption{Przykładowy schemat działania REST API~\cite{SchematRest}}\label{rys:rest}
\end{figure}

Zapytania zazwyczaj wysyłane są przy pomocy protokołu HTTP. Zarówno do zapytań, jak i do odpowiedzi, najczęściej wykorzystywane są metody GET oraz POST. Informacje przesyłane między klientem a serwerem muszą być precyzyjne, zwykle występują one w formacie JSON. 

\subsection{Modele}
Jednym z najważniejszych mechanizmów zaimplementowanych w Django są Modele. Są one swego rodzaju mapowaniem klas języka Python na tabele bazy danych. Takie rozwiązanie zapewnia bardzo szybki i wygodny dostęp do przechowywanych informacji. Klasy te zdefiniowane są w pliku "models.py". Wykorzystując wbudowane funkcje wykorzystywanej platformy programistycznej, możemy zarówno wygenerować tabele bazy danych na podstawie modeli, jak i wygenerować modele na podstawie gotowych tabel.

W projekcie znajduje się następujące siedem modeli: 
\begin{itemize}
	\item Classrooms
	\item Lessons
	\item Planners
	\item Polls
	\item Subjects
	\item Teachers
	\item Timetables
\end{itemize}
Każdy odpowiada jednej tabeli z bazy danych, pola w każdej z klas są analogiczne do kolumn w odpowiednich tabelach. Zawierają informacje takie jak mp. typ danych, maksymalna długość oraz co jest kluczem głównym.
Implementacja przykładowej klasy:
\begin{figure}[H]
	\centering\includegraphics[width=\textwidth]{figures/TeachersModel}
	\caption{Implementacja klasy Teachers}\label{rys:TeachersModel}
\end{figure}

\subsection{Adresy internetowe}
Kolejnym elementem implementacji są adresy internetowe. W pliku "urls.py" określone zostały wykorzystywane w projekcie adresy URL, oraz funkcje które mają zostać wywołane w przypadku otrzymania od klienta zapytania na dany adres. Implementacja przedstawiona na rysunku ~\ref{rys:URLs}
\begin{figure}[H]
 	\centering\includegraphics[width=\textwidth]{figures/Urls}
 	\caption{Implementacja adresów URL}\label{rys:URLs}
\end{figure}
W przypadku, jeśli zapytanie przyjdzie na niezadeklarowany adres, zwrócony zostanie komunikat HTTP z kodem 404(Page Not Found).

\subsection{Widoki}
Widoki to nic innego, jak funkcję języka python wykonywane w momencie odbioru zapytania przez serwer. Kiedy klient wyśle zapytanie na dany adres, django sprawdza czy jest on zadeklarowany we wspomnianym wcześniej pliku "urls.py", jeśli tak, to wywoływana jest odpowiednia funkcja z pliku "views.py" wraz z jej parametrem którym jest samo zapytanie. Dzięki takiej parametryzacji, widok ma łatwy dostęp do otrzymanych informacji, sprawdza on czy są one prawidłowe, oraz wykonuje na nich odpowiednie działania. W tym projekcie, w większości przypadków wykonywane są operacje zapisu i odczytu z bazy danych. Do komunikacji wykorzystywane są metody GET oraz POST z protokołu HTTP.
Implementacje przykładowych widoków obsługujacych zapytania dla danych metod HTTP na rysunkach ~\ref{rys:BackendGet} oraz ~\ref{rys:BackendPost}
\begin{figure}[H]
	\centering\includegraphics[width=\textwidth]{figures/BackendGet}
	\caption{Widok obsługujący zapytanie typu GET}\label{rys:BackendGet}
\end{figure}
\begin{figure}[H]
	\centering\includegraphics[width=\textwidth]{figures/BackendPost}
	\caption{Widok obsługujący zapytanie typu POST}\label{rys:BackendPost}
\end{figure}

\section{Główne funkcjonalności}
\subsection{Rejestracja i logowanie}
Rejestracja oraz logowanie zrealizowane są za pomocą dwóch adresów, co za tym idzie również dwóch widoków, obsługujących zapytania z metodą POST. Do zarejestrowania nowego użytkownika wymagane jest podanie danych takich informacji jak adres e-mail, login oraz hasło. Funkcja obsługująca rejestrację zapisuje te dane do odpowiedniej tabeli, pod warunkiem że dany użytkownik nie istnieje już w bazie danych, a nastęnie zwraca odpowiedni komunikat. W przypadku logowania wymagane jest podanie adresu e-mail oraz hasła podanych przy rejestracji. Jeśli podane dane są poprawne, generowany oraz zwracany klientowi jest token(JWT). Zakodowane w nim są wszystkie dane podane przy rejestracji, co pozwala na identyfikacje użytkownika przy dalszej komunikacji. Generacja JWT na rysunku ~\ref{rys:jwt} 
\begin{figure}[H]
	\centering\includegraphics[width=\textwidth]{figures/jwt}
	\caption{Fragment kodu generujący JWT}\label{rys:jwt}
\end{figure}
\subsection{Uzupełnianie danych potrzebnych do generacji planu}
Dane które są wymagane do generacji planu lekcji, można podzielić na 4 grupy:
\begin{itemize}
	\item Nauczyciele
	\item Klasy
	\item Sale lekcyjne
	\item Przedmioty
\end{itemize}
w projekcie zaimplementowane są widoki, pozwalające dodawać, edytować, usuwać oraz zwracać informacje o każdej z tych grup. W większości przypadków są to proste operacje zapisu i odczytu z bazy danych, jednak w przypadku dodawania informacji o nauczycielach istnieje dodatkowa funkcjonalność, pozwalająca na wysłanie do nich wiadomości e-mail z adresem do indywidualnej ankiety, w której można określić preferowane godziny pracy. Przy pomyślnej realizacji zapytania o wysłanie wiadomości e-mail, w bazie danych tworzone są wiersze, do których zapisane będą dane z ankiet. Są one identyfikowane po unikatowym numerze, przekazywanym jako parametr w adresie URL. Dzięki takiemu rozwiązaniu, generowany jest indywidualny link do ankiety dla każdego nauczyciela.
\subsection{Generacja oraz zwracanie planów}
Generacja oraz zwracanie planów zrealizowane są za pomocą czterech widoków. Funkcja generacji planu, ma za zadanie odczytać wszystkie potrzebne informacje, i w odpowiednim formacie przekazać je do generatora. Ze względu na to że klient nie powinien oczekiwać na odpowiedź oraz że czas generacji planu może być stosunkowo długi, widok ten w przypadku pomyślnego przygotowania danych wejściowych, zwraca informację o rozpoczęciu generacji planu, nie czekając na zakończenie procesu. Dzięki temu, proces generacji nie zaburza połączenia klient-serwer. Kiedy generator skończy prace, plany lekcji zapisywane są w bazie danych w trzech wersjach, z perspektywy: klas, nauczycieli oraz sal lekcyjnych.
Każdy z tych planów, jest zwracany za pomocą pozostałych trzech widoków.

\section{Baza danych MySQL}