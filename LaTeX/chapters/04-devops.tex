
\chapter{Metodyka pracy oraz przygotowanie infrastruktury informatycznej}

\section{Wstęp}
Projekt powstawał w metodyce DevOps. Takie podejście pozwoliło na szybsze dostarczenie finalnego produktu. Wysoki poziom kooperacji wynikający z metodki DevOps pozwolił na zmniejszenie kosztów dostarczenia produktu oraz znaczne zwiększenie jego spójności. Potrzebna jest jednak mocno rozwinięta infrastruktura informatyczna służąca podtrzymaniu DevOps lifecycle.

\section{Pojęcia}
	\subsection{DevOps}
	DevOps~\cite{devops} to zestaw praktyk, narzędzi i filozofii kulturowej, które automatyzują i integrują procesy pomiędzy zespołami programistów oraz IT. Kładzie nacisk na wzmocnienie pozycji zespołu, komunikację i współpracę między zespołami oraz automatyzację technologii.
	Ruch DevOps rozpoczął się około 2007 roku, kiedy społeczności programistów i operatorów IT wyraziły zaniepokojenie tradycyjnym modelem rozwoju oprogramowania, w którym programiści piszący kod pracowali oddzielnie od operatorów, którzy wdrażali i wspierali kod. Termin DevOps, będący połączeniem słów \textit{development} i \textit{operations}, odzwierciedla proces integracji tych dyscyplin w jeden, ciągły proces.
	\subsection{Continous Integration}
	\textit{Continous Integration and Continous Delivery} (CI/CD)~\cite{ci} to metoda częstego dostarczania aplikacji do klientów poprzez wprowadzenie automatyzacji do etapów tworzenia aplikacji. Główne pojęcia przypisane do CI/CD to ciągła integracja, ciągłe dostarczanie i ciągłe wdrażanie. CI/CD jest rozwiązaniem problemów, jakie integracja nowego kodu może powodować dla zespołów programistycznych i operacyjnych.
	W szczególności, CI/CD wprowadza ciągłą automatyzację i ciągłe monitorowanie w całym cyklu życia aplikacji, od fazy integracji i testowania po dostarczanie i wdrażanie. Łącznie, te połączone praktyki są często określane jako CI/CD i są wspierane przez zespoły programistów i operatorów pracujących razem zpodejściem DevOps lub SRE (\textit{site reliability engineering}).
	\subsection{Kontrola wersji}
	Kontrola wersji~\cite{version_control}, znana również jako kontrola źródła, jest praktyką śledzenia i zarządzania zmianami w kodzie oprogramowania. Systemy kontroli wersji to narzędzia programowe, które pomagają zespołom programistów zarządzać zmianami w kodzie źródłowym w czasie.


\section{Narzędzia i technologie}
	\subsection{Amazon Web Services}
	\textit{Amazon Web Services} (AWS)~\cite{aws} jest spółką zależną firmy Amazon, dostarczającą platformy chmury obliczeniowej na żądanie oraz interfejsy API osobom prywatnym, firmom i rządom na zasadzie \textit{pay-as-you-go}. Te usługi internetowe w chmurze obliczeniowej zapewniają różnorodne podstawowe abstrakcyjne elementy infrastruktury technicznej oraz narzędzia i bloki do obliczeń rozproszonych. Jedną z tych usług jest \textit{Amazon Elastic Compute Cloud} (EC2), która pozwala użytkownikom mieć do dyspozycji wirtualny klaster komputerów, dostępny przez cały czas, przez Internet. Wirtualne komputery AWS emulują większość atrybutów prawdziwego komputera, w tym sprzętowe jednostki centralne (CPU) i procesory graficzne (GPU) do przetwarzania danych, pamięć lokalną/RAM, pamięć masową HDD/SSD, wybór systemów operacyjnych, sieci oraz wstępnie załadowane oprogramowanie użytkowe, takie jak serwery internetowe, bazy danych i zarządzanie relacjami z klientami (CRM).
	
	\subsection{Git}
	Git~\cite{git} to darmowe narzędzie open-source służące do kontroli wersji, zaprojektowane do obsługi wszystkiego, od małych do bardzo dużych projektów z dużą prędkością i wydajnością.	
	
	\subsection{GitHub}
	GitHub~\cite{github} jest dostawcą hostingu internetowego dla rozwoju oprogramowania i kontroli wersji przy użyciu Git. Oferuje on funkcje rozproszonej kontroli wersji i zarządzania kodem źródłowym (SCM) Git, a także własne funkcje.
	
	\subsection{CircleCI}
	CircleCI~\cite{circleci} jest platformą obsługującą \textit{Continous Integration and Continous Delivery} (CICD), która pomaga zespołom programistycznym szybko i pewnie wypuszczać kod poprzez automatyzację procesu budowania, testowania i wdrażania. Pozwala to zespołom szybko się rozwijać, łatwo skalować i budować spójne produkty.

\section{Przygotowanie infrastruktury informatycznej}
	

